\section{Jakub Wądrzyk}
\label{wadrjaku}
To jest rozdział poświęcony kotom \\

Ale najpierw matematyka. Wszyscy kochamy matme:
\begin{displaymath}
    \centering
    x_{1,2} = \frac{-b \pm  \sqrt{\Delta } }{2a} , \Delta \geq 0 
\end{displaymath}

Tabela \ref{tab:even&odd}
\begin{table}[htbp]
    \centering
    \begin{tabular}{c||l|l|}
            &  even & odd   \\ \hline \hline
        1   & 2     & 5     \\ \hline
        2   & 78    & 55    \\ \hline
        3   & 124   & 227   \\ \hline
        4   & 519   & 919   \\ \hline
        5   & 3468  & 2137  \\
    \end{tabular}
    \caption{Bardzo potrzebne dane}
    \label{tab:even&odd}
\end{table}

%Tak 519 nie jest parzyste

\textbf{Kot ma:}
\begin{itemize}
    \item [-] cztery łapy
    \item [!] jeden ogon
    \item [>] około dwieście pięćdziesiąt kości
    \item i wciąż nie wiemy jak mruczy
\end{itemize}

\pagebreak

Zdjęcie \ref{fig:kot}:
\begin{figure}[htbp]
    \centering
    \includegraphics[width=0.5\textwidth]{pictures/kot.png}
    \caption{Właśnie tak wygląda kot.}
    \label{fig:kot}
\end{figure}

\textbf{Kot domowy} (\textit{Felis catus}) – udomowiony gatunek \emph{ssaka} z rzędu \emph{drapieżnych} z rodziny \emph{kotowatych}. \textbf{Koty} zostały udomowione około \texttt{9500} lat temu i są obecnie \textsc{najpopularniejszymi {\em zwierzętami domowymi} na świecie}. Gatunek ten \underline{prawdopodobnie} pochodzi od \texttt{kota nubijskiego}, przy czym w \textit{Europie {\em krzyżował się ze} żbikiem}.