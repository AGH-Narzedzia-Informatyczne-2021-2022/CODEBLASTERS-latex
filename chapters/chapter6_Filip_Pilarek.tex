\section{Filip Pilarek}
\label{wadrjaku}
Tym razem coś o crypto \\

polecam mathpix!
Pokażmy coś matematycznie
\begin{displaymath}
    \centering
        \log _{b} c=\frac{\log _{a} c}{\log _{a} b}
\end{displaymath}

\pagebreak

Tabela \ref{tab:random}
\begin{table}[htbp]
\centering
\begin{tabular}{|c|c|c|c|c|}
\hline
15  & 2323 & 32  & 23   & 23   \\ \hline
22  & 23   & 32  & 2123 & 32   \\ \hline
23  & 1    & 616 & 231  & 2321 \\ \hline
242 & 12   & 123 & 21   & 321  \\ \hline
\end{tabular}
\caption{Losowe liczby}
\label{tab:random}
\end{table}

\begin{itemize}
  \item piękna
  \item dziś 
  \item[-] pogoda
  \item[MONEYY] by kupić
  \item[] kryptowalute
\end{itemize}

\vspace{14pt}

\begin{enumerate}
  \item polecam
  \item bardziej
  \item[!] niż
  \item[NOTE] giełde gpw
\end{enumerate}


Zdjęcie \ref{fig:crypto}:
\begin{figure}[htbp]
    \centering
    \includegraphics[width=0.5\textwidth]{pictures/crypto.jpg}
    \caption{Piękne monety.}
    \label{fig:crypto}
\end{figure}

\textbf{Kryptowaluta} (\textit{crypto currency}) – rozproszony system księgowy \emph{bazujący} na kryptografii \emph{przechowujący informację o stanie posiadania w umownych jednostkach.}  \emph{Stan posiadania związany jest z poszczególnymi węzłami systemu (\textbf{„portfelami”}) w taki sposób}. aby kontrolę nad danym portfelem miał wyłącznie \textit{posiadacz odpowiadającego} mu klucza \underline{prywatnego} i niemożliwe było dwukrotne wydanie tej samej \textsc{jednostki}. \texttt{Kryptowaluta} jest szczególnym przypadkiem waluty wirtualnej.