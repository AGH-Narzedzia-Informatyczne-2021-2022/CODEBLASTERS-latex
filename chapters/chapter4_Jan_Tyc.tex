\section{Jan Tyc}
\label{tycjan}

Oto rozdział o szachach \\
 A teraz rówanie opisujące ilość możliwych różnych parti szachowych
 \begin{math}\alpha = 10^{120}\end{math}
 A to wzór na całke zupełnie bez powodu
 \[ \int x^n \ dx = \frac{1}{n+1} x^{n+1} + C\]
 
 
  
 
 Tabela sławnych szachistów
 
 \ref{tab:szachy}
 \begin{table}[htbp]
\begin{tabular}{|l|l|l|ll}
\cline{1-3}
\textbf{Imię i nazwisko} & \textbf{ELO} & \textbf{Kraj} &  &  \\ \cline{1-3}
Magnus Carlsen           & 2855         & Norwegia      &  &  \\ \cline{1-3}
Jan Krzysztof-Duda       & 2759         & Polska        &  &  \\ \cline{1-3}
Alireza Firouzja         & 2781         & Francja       &  &  \\ \cline{1-3}

\label{tab:szachy}
\end{tabular}
\end{table}
 
\pagebreak
 
 Figury szachowe to:
\begin{enumerate}
    \item Król
    \item Hetman
    \item Wieża
    \item Goniec
    \item Koń
    \item Pion
\end{enumerate}

Zdjęcie dwóch koni \ref{fig:szachy}:
\begin{figure}[htbp]
    \centering
    \includegraphics[width=0.5\textwidth]{pictures/szachy.jpg}
    \caption{Szachy}
    \label{fig:szachy}
\end{figure}


\textbf{Szachy to gra królów.} Początki siegają wielu setek lat wstecz. Jednak pierwsze mistrzostwa rozegrano dopiero w \underline{XIX wieku}. Szachy pomagają rozwijąć {\huge Logikę} oraz {\footnotesize Pamięć}. \texttt{Wymagają wiele czasu}, ale dają dużo \textsc{satysfakcji}.


