\section{Maciej Adamus}
\label{macadam}

Tak wygląda pączek

\begin{figure}[htbp] 
    \centering
    \includegraphics[width=0.4\textwidth]{pictures/pączek.jpg} 
    \caption{Pączek}
    \label{fig:pączek}
\end{figure}

Tabela \ref{tab:prime_numbers}:
\begin{table}[htbp]
\centering
\begin{tabular}{|l||l|l|l|}
\hline
lp & p  & $p^2$ & $p^3$ \\ \hline \hline
1  & 2  & 4                    & 8                    \\ \hline
2  & 3  & 9                    & 27                   \\ \hline
3  & 5  & 25                   & 125                  \\ \hline
4  & 7  & 49                   & 343                  \\ \hline
5  & 11 & 121                  & 1331                 \\ \hline
\end{tabular}
\label{tab:prime_numbers}
\caption{Liczby pierwsze i ich potęgi}
\end{table}


Przykładowe równanie matematyki alternatywnej 
\begin{displaymath}
    \frac{sin x}{x}=six=6
\end{displaymath} 



Jak zrobić pączka
\begin{enumerate}
    \item zrób ciasto
    \item umieść nadzienie w środku
    \item usmaż
    \item posyp cukrem pudrem
\end{enumerate}

Czym można nadziać pączka
\begin{itemize}
    \item[-] marmolada
    \item[-] nutella
    \item[-] budyń
\end{itemize}

\textbf{\underline{Pączek}} w kuchni polskiej to wyrób \underline{w postaci ciasta} drozdżowego z mąki pszennej uformowanej w kształt lekko \emph{spłaszczonej} \textbf{\emph{kuli}} i usmażonego na głębokim tłuszczu (najczęściej \textbf{smalcu}) na kolor ciemnozłoty (patrz rysunek \ref{fig:pączek}). \par
Pączki są szcególnie popularne w \textbf{tłusty czwartek}, czyli ostatni czwartek przed \emph{wielkim postem}.


